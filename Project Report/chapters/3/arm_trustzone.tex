\section{ARM TrustZone}

\subsection{An introduction to ARM TrustZone}

ARM TrustZone is a method used to separate “secure” applications from non-secure ones in two separate worlds, memory can be designated as secure or not secure. Execution is transferred from each world either through interrupts or direct function calls. This page describes the memory aspect quite well. While this stack overflow q and a explores it in a more real life example. This is very well described by Ngabonziza et al \cite{Ngabonziza2017}.

I believe secure world TrustZone applications should have access to normal world memory, this will enable static attestation.

Provisioning is an interesting aspect which I haven’t quite worked out out.

A secure world application would need to supply the normal world application with a nonce to make sure the response is not a replay.