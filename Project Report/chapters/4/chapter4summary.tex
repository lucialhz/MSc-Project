\section{Chapter Summary}
Control-flow integrity and attestation schemes differ greatly in their construction. Some monitor transitions by making changes to the instructions contained within an application, where the new instructions call the monitoring code whenever a branch occurs. Some encrypt instructions to be executed, with these only being successfully decrypted if the previously executing instructions were the ones expected. Other solutions make use of shadow stacks to ensure that the destination of backwards edge transitions (e.g. function returns) match the locations from which they were called.

Using comparison tables~\ref{tab:solutionComparisonRequirements},~\ref{tab:solutionComparisonProperties1} and~\ref{tab:solutionComparisonProperties2}, it can be identified that all of the solutions are compatible with some form embedded systems, and the solutions which implement additional hardware do not require the inclusion of a trusted platform module (TPM), although some require secure storage of cryptographic keys. Comparing the overhead of each solution is not a good comparison factor, as each solution has been implemented on different equipment. With that being said it does make for some interesting reading, as the range of overhead varies dramatically. LO-FAT \cite{Dessouky2017} achieves an impressively low 0\% execution time overhead (thanks to offloading the attestation mechanism to additional hardware), while SEP \cite{Lee2019} incurs an overhead of 235\% (due to the padding of instructions and slow implementation of its decryption function). Aside from these extremes, most of the other schemes incur up to 5\% overhead, which makes them within the range of adoption (according to \cite{Szekeres2013}).

Static attestation schemes follow much more of a pattern (in comparison to CFI attestation), where a hash is produced by taking samples of the entire memory space, or samples of application code stored in memory. The approaches differ in that some also carry out a self-attestation stage on the instructions which carries out the attestation process, with the results of this being time critical to ensure the initial check has not been run through a proxy.

The comparison table~\ref{tab:staticAttestationComparisonRequirements} containing static attestation shows just how similar each scheme is with one another, with only SMART \cite{For} differing significantly due to its requirement for hardware modification.