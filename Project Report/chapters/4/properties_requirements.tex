\section{Introduction}\label{requirementsIntro}

We will consider each solution described in \ref{comparisonOfCFISolutions} against a set of requirements closely resembling those defined in \ref{projectRequirements} and present the results in table ~\ref{tab:solutionComparisonRequirements}.

\section{Requirements}\label{Requirements}

\begin{enumerate}
	\item Works with compiled binaries: software does not need to be recompiled to be compatible with the solution.
	\item Works with external libraries: libraries can be used without need access to their source code.
	\item Can be used to bind software and hardware: would allow for future functionality of binding software to hardware.
	\item Works with embedded systems
	\item Immediate identification: whether the solution can identify a breach in CFI as soon as it occurs.
	\item Not reliant on TPM: Embedded systems are resource-constrained and therefore may not have TPMs.
\end{enumerate}

\section{Properties}\label{Properties}

\begin{enumerate}
	\item Hardware based: The solution is implemented through hardware.
	\item Software based: The solution is implemented through software.
	\item Hardware modification required: This needs to be removed from the table as is the same as property 1.
	\item Source code modification required: Source code needs to be changed for implementation.
	\item Granularity: How granular the control-flow is examined, e.g. individual instructions or basic blocks.
	\item Provisioning method: How software is updated.
	\item How it is secured?: What is providing the security.
	\item CPU Overhead
	\item Storage overhead
	\item Memory overhead
	\item Execution time overhead
	\item Compatible architectures
	\item Action after identification of insecure event
	\item Prevention / Detection / Attestation
	\item CFG or other: Whether the solution makes use of CFGs
\end{enumerate}

\ifnotesincluded
\section{Conclusion}
\fi
