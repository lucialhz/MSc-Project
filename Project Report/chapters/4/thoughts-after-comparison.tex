\section{Thoughts after comparison}

\subsection{Signature Modelling}

When first introduced to CFI, it was described in relation to control-flow graphs (CFGs). Here if a series of executed instructions do not follow the trace of the program’s CFG the control-flow integrity has been lost. The paper \cite{Yang2013} did a very good job at describing this method as implemented in software. They also performed a thorough analysis of various existing solutions. This methodology was implemented in hardware in the paper written by Werner et al \cite{Werner2016}.

A problem with signature modelling is that the combination sequences of instructions (and their accompanying signatures) accumulate very quickly, requiring more storage if they are to be compared to. This is a particular problem when considering embedded systems due to their inherent limitation on resources. Another problem is the necessity to compute signatures in real-time, this would add a fair amount of processing overhead. Although, as we have seen in other methods \cite{Davi2015} \cite{Lee2019}, decryption is a common method of ensuring CFI, so perhaps this is not an unreasonable expectation.

\subsection{Shadow Stacks}
Some \cite{Christoulakis2016} have argued that ensuring CFI without the use of shadow stacks (SSs) is an impossible task and that “the use of a a shadow stack is mandatory for any practical CFI deployment”. A shadow stack is basically what is says on the tin. A separate stack is stored which contains just the return addresses. This solves the problem of ROP, especially when working with commonly called or system functions as the implementation of the shadow stack ensures the calling function has to be the return function.

There are some flaws with shadow stacks which include:
\begin{itemize}
	\item Storage of shadow stacks can be cumbersome - especially if they are to be secure (as they would have to be if memory access is an assumed capability of an attacker).
	\item Maintaining a shadow stack in multi-threaded systems will be troublesome - searching through the SS for a process ID will add additional time overhead to any processing.
	\item Recursive functions can add multiple entries of the same return address to the SS, this is an unnecessary addition so is reduced down to a single it flag (or counter) by some solutions. This can raise the problem of manipulating the number of recursions by an attacker, however it is argued by some that this is a very limited attack.
	\item Instructions such as \verb|longjmp| will skip back through multiple return addresses, so this needs to be taken into account - either by removing the instruction from the ISA or another method such as iterating through the SS until the return address is identified (with the above addresses then removed from the SS to maintain a proper representation of the stack).
\end{itemize}