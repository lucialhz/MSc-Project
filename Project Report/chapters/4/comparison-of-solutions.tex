\section{Comparison of solutions} \label{comparisonOfCFISolutions}

\subsection{Established Criteria}\label{requirementsIntro}

We will consider each solution described in \ref{comparisonOfCFISolutions} against a set of requirements closely resembling those defined in \ref{projectRequirements} and present the results in table ~\ref{tab:solutionComparisonRequirements}.

\section{Requirements}\label{Requirements}

\begin{enumerate}[label=(\arabic*),noitemsep]
	\item Works with compiled binaries: software does not need to be recompiled to be compatible with the solution.
	\item Works with external libraries: libraries can be used without the need to access their source code.
	\item Can be used to bind software and hardware: would allow for future functionality of binding software to hardware.
	\item Works with embedded systems
	\item Immediate identification: whether the solution can identify a breach in CFI as soon as it occurs.
	\item Not reliant on TPM: Embedded systems are resource-constrained and therefore may not have TPMs.
\end{enumerate}

\section{Properties}\label{Properties}

\begin{enumerate}[label=(\arabic*),noitemsep]
	\item Hardware based: The solution is implemented through hardware.
	\item Software based: The solution is implemented through software.
	\item Source code modification required: Source code needs to be changed for implementation.
	\item Granularity: How granular the control-flow is examined, e.g. individual instructions or basic blocks.
	\item Provisioning method: How software is updated.
	\item How it is secured?: What is providing the security.
	\item CPU Overhead
	\item Storage overhead
	\item Memory overhead
	\item Execution time overhead
	\item Compatible architectures
	\item Action after identification of a CFI violation
	\item Prevention / Detection / Attestation
	\item CFG or other: Whether the solution makes use of CFGs
\end{enumerate}

\subsubsection{Introduction}

Here we will provide an in-depth review of a number of control-flow integrity and control-flow attestation mechanisms.

\subsubsection{C-FLAT}
C-FLAT uses ARM TrustZone to facilitate control-flow monitoring and attestation \cite{Abera2016}. Binaries are re-factored to enable normal-world programs to log control-flow changes with a measurement engine operating within the ``secure world'' (via trampolines set up in the ``normal world''). Control-flow edges are hashed together to form a hash of the complete control flow. Loops are handled as their own sub-program which require separate meta data to be gathered and included in the attestation report. This solution requires code instrumentation but also requires programs to not use \verb|LR| as a general purpose register, where the use of the \verb|-ffixed-lr| compilation option in GCC\footnote{\url{https://gcc.gnu.org/onlinedocs/gcc-4.6.1/gcc/Global-Reg-Vars.html}} is recommended, for example.

\subsubsection{LO-FAT}
LO-FAT implements a hardware-based solution which monitors control-flow instructions by collecting the program counter and instruction executed for each clock cycle, then filtering out branch, jump and return instructions \cite{Dessouky2017}. LO-FAT follows a similar method as C-FLAT where a hash is built up of each control-flow operation, along with special meta data for loops. It uses additional hardware components (designed onto an FPGA - Virtex-7 XC7Z020) to operate alongside the main processing unit. Through the use of these components (almost co-processors) the solution is able to run alongside normal operations, therefore removing the burden of additional instruction executions and the requirement for software instrumentation (a process of changing the contents of software to meet the requirements of the solution).

\subsubsection{Secure-Execution Processor (SEP)}
SEP uses encryption of each instruction which takes the previously executed instruction as an input to the decryption algorithm \cite{Lee2019}. Decryption is performed within the processor pipeline which prevents the malicious diversion of control-flow, if an attempt is made to alter the control-flow the processor will attempt to execute a garbage instruction, the result of which will be to hit the 'kill switch'. The solution goes into great detail on how to deal with particularly troublesome instructions. Unfortunately due to the large memory and execution overhead (235\%), the limited instruction set and the use of a slow encryption mechanism, this solution still has a long way to go before it can be properly considered, though the principal and elegance of the solution is promising. This solution requires code instrumentation in order to deterministically encrypt each instruction.

\subsubsection{CCFI-Cache}
CCFI-Cache implements a hardware-based solution where control-flow metadata (including the number of instructions in the BBL, the valid destination addresses for the current BBL and a hash value of the instructions of the contained within the BBL)is fetched by the cache (CCFI-cache) and checked by the checker (CCFI-checker) \cite{Danger2018}. The metadata and BBLs are padded out to ensure that their length matches (empty regions for the metadata and \verb|NOP|s for the BBL), this enables the checker to know if either the metadata or BBL has been modified in a manner which affects the length. The checker checks that the destination of a BBL matches the pre-computed metadata and that the hashed instructions within the BBL match the stored hash. It also utilises a shadow stack to ensure the correctness of return addresses. If a violation is detected an interrupt on the CPU will be triggered. In this author's opinion this solution provides the best coverage for protecting control-flow integrity. This solution requires code instrumentation.

\subsubsection{HCFI}
HCFI implements a shadow-stack on additional hardware \cite{Danger2018}. The paper does not describe how it ensures the integrity of forward edges but it does provide a thorough analysis of backward-edge protection through the use of a shadow stack. It is not clear if this solution requires code instrumentation or not.

\subsubsection{HAFIX}
HAFIX (like HCFI \cite{Danger2018}) focuses on backward-edge protection through their implementation of a shadow stack \cite{Davi2015}. HAFIX and HCFI have different approaches with regards to handling recursive functions and loops. This solution requires code instrumentation.

\subsubsection{SOFIA}
SOFIA works in a similar manner to SEP \cite{Lee2019}, where instructions are decrypted using program counters of previously executed instructions \cite{Davi2015}. SOFIA handles instructions with multiple callers by creating a multiplexor block, which allows two predecessors, so multiple predecessors can  be used by creating layers of multiplexor blocks. The fast expansion of multiplexor blocks could prove problematic in terms of size and operating time of applications. The solution requires code instrumentation (for creating the multiplexor blocks and encryption of instructions).

\subsubsection{Strategy without tactics (Sullivan)}
In Strategy without tactics \cite{Davi2015}, the author uses trampolines to deal with multiple callers, where a single destination is replaced with a trampoline. Through adding a trampoline the solution is able to transform a call source\slash destination pair into a unique call source\slash trampoline pair which solves the problem of basic blocks with a number of possible predecessors. From the trampoline, a direct jump is issued to the original destination. A label state stack records backward-edges and a label state register records forward-edges. 

\subsubsection{Learnings}
It has been noted on several occasions that backward edges cannot be fully protected without the use of shadow stacks, as returns may be legal for many functions (e.g. \verb|printf|). Keeping an audit record of control-flow would eliminate the requirement for a shadow stack as the historic tracking of the control-flow would indicate if a backward edge had been exploited.