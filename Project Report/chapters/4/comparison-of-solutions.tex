\section{Comparison of Solutions Providing Control-Flow Integrity and Attestation} \label{comparisonOfCFISolutions}

\subsection{Established Criteria}\label{requirementsIntro}

We will consider each solution described in \ref{comparisonOfCFISolutions} against a set of requirements closely resembling those defined in \ref{projectRequirements}. These requirements have been decided based on their relevance to this project and as a method to different each solution from one another.

\subsubsection*{Requirements}\label{Requirements}

The requirements included as established criteria are:

\begin{enumerate}[label=(\arabic*),noitemsep]
	\item Works with compiled binaries: software does not need to be recompiled to be compatible with the solution.
	\item Works with external libraries: libraries can be used without the need to access their source code.
	\item Can be used to bind software and hardware: would allow for future functionality of binding software to hardware.
	\item Works with embedded systems
	\item Immediate identification: whether the solution can identify a breach in CFI as soon as it occurs.
	\item Not reliant on TPM: Embedded systems are resource-constrained and therefore may not have TPMs.
\end{enumerate}

\subsubsection*{Properties}\label{Properties}

The properties of each solution which are of particular interest are:

\begin{enumerate}[label=(\arabic*),noitemsep]
	\item Hardware based: The solution is implemented through hardware.
	\item Software based: The solution is implemented through software.
	\item Source code modification required: Source code needs to be changed for implementation.
	\item Granularity: How granular the control-flow is examined, e.g. individual instructions or basic blocks.
	\item Provisioning method: How software is updated.
	\item How it is secured?: What is providing the security.
	\item CPU Overhead
	\item Storage overhead
	\item Memory overhead
	\item Execution time overhead
	\item Compatible architectures
	\item Action after identification of a CFI violation
	\item Prevention\slash Detection\slash Attestation
	\item CFG or other: Whether the solution makes use of CFGs
\end{enumerate}

\subsection{Solutions Presented}

The following solutions have been selected for comparison due to their prominence in research literature and their relevance to the subject of this project. We will provide a brief description of each solution prior to presenting them in a table against the requirements and properties defined in~\ref{requirementsIntro}.

\subsubsection*{C-FLAT}
C-FLAT uses ARM TrustZone to facilitate control-flow monitoring and attestation \cite{Abera2016}. Binaries are re-factored to enable normal-world programs to log control-flow changes with a measurement engine operating within the ``secure world'' (via trampolines set up in the ``normal world''). Control-flow edges are hashed together to form a hash of the complete control flow. Loops are handled as their own sub-program which require separate meta data to be gathered and included in the attestation report. This solution requires code instrumentation but also requires programs to not use \verb|LR| as a general purpose register, where the use of the \verb|-ffixed-lr| compilation option in GCC\footnote{\url{https://gcc.gnu.org/onlinedocs/gcc-4.6.1/gcc/Global-Reg-Vars.html}} is recommended, for example.

\subsubsection*{LO-FAT}
LO-FAT implements a hardware-based solution which monitors control-flow instructions by collecting the program counter and instruction executed for each clock cycle, then filtering out branch, jump and return instructions \cite{Dessouky2017}. LO-FAT follows a similar method as C-FLAT where a hash is built up of each control-flow operation, along with special meta data for loops. It uses additional hardware components (designed onto an FPGA - Virtex-7 XC7Z020) to operate alongside the main processing unit. Through the use of these components (almost co-processors) the solution is able to run alongside normal operations, therefore removing the burden of additional instruction executions and the requirement for software instrumentation (a process of changing the contents of software to meet the requirements of the solution).

\subsubsection*{Secure-Execution Processor (SEP)}
SEP uses encryption of each instruction which takes the previously executed instruction as an input to the decryption algorithm \cite{Lee2019}. Decryption is performed within the processor pipeline which prevents the malicious diversion of control-flow, if an attempt is made to alter the control-flow the processor will attempt to execute a garbage instruction, the result of which will be to hit the 'kill switch'. The solution goes into great detail on how to deal with particularly troublesome instructions. Unfortunately due to the large memory and execution overhead (235\%), the limited instruction set and the use of a slow encryption mechanism, this solution still has a long way to go before it can be properly considered, though the principal and elegance of the solution is promising. This solution requires code instrumentation in order to deterministically encrypt each instruction.

\subsubsection*{CCFI-Cache}
CCFI-Cache implements a hardware-based solution where control-flow metadata (including the number of instructions in the BBL, the valid destination addresses for the current BBL and a hash value of the instructions of the contained within the BBL)is fetched by the cache (CCFI-cache) and checked by the checker (CCFI-checker) \cite{Danger2018}. The metadata and BBLs are padded out to ensure that their length matches (empty regions for the metadata and \verb|NOP|s for the BBL), this enables the checker to know if either the metadata or BBL has been modified in a manner which affects the length. The checker checks that the destination of a BBL matches the pre-computed metadata and that the hashed instructions within the BBL match the stored hash. It also utilises a shadow stack to ensure the correctness of return addresses. If a violation is detected an interrupt on the CPU will be triggered. In this author's opinion this solution provides the best coverage for protecting control-flow integrity. This solution requires code instrumentation.

\subsubsection*{HCFI}
HCFI implements a shadow-stack on additional hardware \cite{Danger2018}. The paper does not describe how it ensures the integrity of forward edges but it does provide a thorough analysis of backward-edge protection through the use of a shadow stack. It is not clear if this solution requires code instrumentation or not.

\subsubsection*{HAFIX}
HAFIX (like HCFI \cite{Danger2018}) focuses on backward-edge protection through their implementation of a shadow stack \cite{Davi2015}. HAFIX and HCFI have different approaches with regard to handling recursive functions and loops. This solution requires code instrumentation.

\subsubsection*{SOFIA}
SOFIA works in a similar manner to SEP \cite{Lee2019}, where instructions are decrypted using program counters of previously executed instructions \cite{Davi2015}. SOFIA handles instructions with multiple callers by creating a multiplexor block, which allows two predecessors, so multiple predecessors can  be used by creating layers of multiplexor blocks. The fast expansion of multiplexor blocks could prove problematic in terms of size and operating time of applications. The solution requires code instrumentation (for creating the multiplexor blocks and encryption of instructions).

\subsubsection*{Strategy Without Tactics (Sullivan)}
In Strategy Without Tactics \cite{Davi2015}, the author uses trampolines to deal with multiple callers, where a single destination is replaced with a trampoline. Through adding a trampoline the solution is able to transform a call source\slash destination pair into a unique call source\slash trampoline pair which solves the problem of basic blocks with a number of possible predecessors. From the trampoline, a direct jump is issued to the original destination. A label state stack records backward-edges and a label state register records forward-edges. 

\subsection{Discussion of Findings}
\subsubsection*{Comparison Against Requirements}
\begin{table}
\centering{
\scalebox{0.6}{
\begin{threeparttable}
\caption{Comparison of Control-flow Integrity and Control-flow Attestation requirements}
\label{tab:solutionComparisonRequirements}
\newcommand*\rot[1]{\hbox to1em{\hss\rotatebox[origin=br]{-60}{#1}}}
\begin{tabular}{l c c c p{3cm} p{8cm} p{3cm}}
\toprule
Scheme & \multicolumn{6}{c}{Requirements}\\
\midrule
 & \rot{(1) Works with compiled binaries}
 & \rot{(2) Works with external libraries}
 & \rot{(3) Can be used for binding}
 & \rot{(4) Works with embedded systems}
 & \rot{(5) Immediate identification}
 & \rot{(6) Not reliant on TPM}\\
\midrule
C-FLAT \cite{Abera2016} & \CIRCLE & \CIRCLE & \Circle & \CIRCLE \raggedright \ ARM Cortex A & \Circle & \Circle\\
LO-FAT \cite{Dessouky2017} & \CIRCLE & \CIRCLE & \Circle & \CIRCLE & \Circle & \CIRCLE\\
SEP-LEE \cite{Lee2019} & \Circle  & \Circle & \CIRCLE & \CIRCLE & \raggedright \CIRCLE \ On jump or return & \CIRCLE\\
CCFI-Cache \cite{Danger2018} & \Circle  & \Circle & \Circle & \CIRCLE & \raggedright \Circle \ End of BBL or number of executions reached & \CIRCLE\\
HCFI \cite{Christoulakis2016} & \CIRCLE  & \CIRCLE & \Circle & \CIRCLE & \CIRCLE & \CIRCLE\\
HAFIX \cite{Davi2015} & \Circle  & \Circle & \Circle & \CIRCLE & \CIRCLE & \CIRCLE\\
SOFIA \cite{DeClercq2017b} & \CIRCLE  & \CIRCLE & \CIRCLE & \CIRCLE & \CIRCLE & \CIRCLE\\
Sullivan \cite{Sullivan2016} & \LEFTcircle  & \CIRCLE & \Circle & \CIRCLE & \CIRCLE & \CIRCLE\\
\bottomrule
\end{tabular}
\begin{tablenotes}
	\item \CIRCLE ~= meets requirement; \LEFTcircle ~= partially meets requirement; \Circle ~= does not meet requirement
\end{tablenotes}
\end{threeparttable}
}}
\end{table}
Table~\ref{tab:solutionComparisonRequirements} shows how (or whether) each solution presented meets the requirements defined in~\ref{requirementsIntro}. It can be seen that all solutions are compatible with embedded systems and most provide immediate identification of violation of CFI. Only two solutions could be used for binding hardware and software, while the mechanisms of operations of the other solutions would not facilitate software\slash hardware binding.

\subsubsection*{Comparison of Properties}
\begin{table}
\newcommand*\rot[1]{\hbox to1em{\hss\rotatebox[origin=br]{-60}{#1}}}
\centering
\scalebox{0.6}{
\begin{threeparttable}
\caption{Comparison of Control-flow Integrity and Control-flow Attestation schemes properties 1-5}
\label{tab:solutionComparisonProperties1}
\begin{tabular}{l c p{3cm} p{5cm} p{5cm} p{5cm}}
\toprule
Scheme & \multicolumn{5}{c}{Properties 1-5} \\
\midrule
 & \rot{(1) Hardware-based}
 & \rot{(2) Software-based}
 & \rot{(3) Source-code modification required} 
 & \rot{(4) Granularity}
 & \rot{(5) Security provided by}\\
\midrule
C-FLAT \cite{Abera2016} & \LEFTcircle \raggedright Star  & \CIRCLE & \raggedright Control flow instructions re-written to go via trampolines & Basic Block & \raggedright\arraybackslash TPM and Digital Signatures (for report)\\
LO-FAT \cite{Dessouky2017} & \CIRCLE & \Circle & \Circle & Basic Blocks & LO-FAT Memory is protected\\
SEP-LEE \cite{Lee2019} & \CIRCLE & \Circle \ Reassembly & \CIRCLE  & Sequential or BBLs & PRINCE\\
CCFI-Cache \cite{Danger2018} & \CIRCLE & \LEFTcircle & \CIRCLE  & \raggedright Sequential (Only number or hash of instructions) & \raggedright\arraybackslash Meta-data in ROM, Hardware Security for controller\\
HCFI \cite{Christoulakis2016} & \CIRCLE & \Circle & \CIRCLE & BBLs & - \\
HAFIX \cite{Davi2015} & \CIRCLE & \Circle & \CIRCLE & Backwards edge & - \\
SOFIA \cite{DeClercq2017b} & \CIRCLE & \Circle & \CIRCLE  & \raggedright Sequential instructions or flows & \raggedright\arraybackslash Encrypt+MAC (RECTANGLE-80)\\
Sullivan \cite{Sullivan2016} & \CIRCLE & \Circle & \CIRCLE & BBLs & \raggedright\arraybackslash Attacker cannot modify code\\
\bottomrule
\end{tabular}
\begin{tablenotes}
	\item \CIRCLE ~= has property; \LEFTcircle ~= partially has property; \Circle ~= does not have property; - = not enough information; star = makes use of hardware features not present in all embedded systems.
\end{tablenotes}
\end{threeparttable}}
\end{table}
\begin{table}
\newcommand*\rot[1]{\hbox to1em{\hss\rotatebox[origin=br]{-60}{#1}}}
\centering
\scalebox{0.6}{
\begin{threeparttable}
\caption{Comparison of Control-flow Integrity and Control-flow Attestation schemes properties 6-14}
\label{tab:solutionComparisonProperties2}
\begin{tabular}{l p{2cm} c p{2cm} p{2cm} p{3cm} p{2cm} p{2cm} p{3cm} p{2cm}}
\toprule
Scheme & \multicolumn{9}{c}{Properties 6-14} \\
\midrule
 & \rot{(6) CPU overhead}
 & \rot{(7) FPGA area overhead}
 & \rot{(8) Memory overhead}
 & \rot{(9) Time overhead}
 & \rot{(10) Compatible architectures}
 & \rot{(11) Action on identification}
 & \rot{(12) Prevention or detection or attestation}
 & \rot{(13) Uses CFG or another method}
 & \rot{(14) Success rate}\\
\midrule
C-FLAT \cite{Abera2016} & - & - & - & 0.03 - 0.13\% & 32-bit RISC & - & Attestation & CFG & -\\
LO-FAT \cite{Dessouky2017} & 0\% & 20\% & - & 0\% & PULPINO RISC-V & - & Attestation & CFG & -\\
SEP-LEE \cite{Lee2019} & - & - & \raggedright 235\% (160 vs 68 operations) & \raggedright 235\% (160 vs 68 operations) & SEP (based on Picoblaze but reduced from 18 bit to 16 bit) & Kill switch & \raggedright Prevention and detection & Sequential or BBL & -\\
CCFI-Cache \cite{Danger2018} & - & 10\% & x2 + 9-30\% for nops & 2-63\% & \raggedright PicoRV (Implementation on RISC-V ISA) & Interrupt & Detection & \raggedright CFG + No of instructions + hash of instructions & \raggedright\arraybackslash Small number of ROP and JOP attacks will still work \\
HCFI \cite{Christoulakis2016} & - & 2.5\% & - & 1-6\% & Leon3\slash SPARC V8 & Interrupt & Detection & \raggedright CFG and Shadow Stack & -\\
HAFIX \cite{Davi2015} & - & 2-8\% & - & 2\% & Intel Siskiyou Peak, SPARC & - & Detection & CFG & 80\%\\
SOFIA \cite{DeClercq2017b} & 13.7\% \raggedright Cycle overhead, 84.6\% Clock speed reduction & 28.2\% & - & 110\% & Leon3\slash SPARC V8 & Processor reset & \raggedright Prevention and detection & \raggedright Sequential and BBLs & -\\
Sullivan \cite{Sullivan2016} & - & 1.78\% & - & 1.75\% & Leon3\slash SPARC V8 & \raggedright Termination of process & Detection & CFG & \raggedright\arraybackslash Dependant on CFG policy (assumed 100\%)\\
\bottomrule
\end{tabular}
\begin{tablenotes}
	\item \CIRCLE ~= has property; \LEFTcircle ~= partially has property; \Circle ~= does not have property; - = not enough information 
\end{tablenotes}
\end{threeparttable}}
\end{table}
Tables~\ref{tab:solutionComparisonProperties1} and~\ref{tab:solutionComparisonProperties2} show how the solutions compare against each other on the basic of some key properties. The majority of solutions use hardware to implement their schemes, and the single one which does not need additional hardware (C-FLAT \cite{Abera2016}) makes use of additional security functionality provided by ARM TrustZone. Most solutions require the alteration of software, be it instrumentation or encryption of instructions. In terms of granularity, half of the solutions operate at a basic block level, while others offer an insight into the actual instructions executing (this functionality is offered thanks to hardware modifications which intercept executing instructions). It is challenging to compare each solution on operating overheads as they have not been tested against the same benchmarks while using the same equipment. FGPA area increase should also be a good measurement however, as noted by \cite{DeClercq2017}, overheads claimed in solutions tend to be referenced against the existing soft-core processor implemented which varies greatly in FGPA area.


