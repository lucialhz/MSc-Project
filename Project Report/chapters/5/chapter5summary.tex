\section{Chapter Summary}
Building a control-flow graph (CFG) is has been an open problem in academia. There are a handful of software packages which aim to offer this functionality (usually as a bi-product of other functionality such as compilation or reverse engineering). The most promising solution, angr \cite{Shoshitaishvili2016}, showed promise, however it only produced single BBL (basic clock) to BBL edges, rather than complete paths. This could be symptom of using CFGs a basis for control-flow integrity (CFI) protection, whereas perhaps generating a collection of all valid paths would provide a superior basis for ensuring secure execution. Based on evaluations taken place, `angr' framework would most likely be the tool used in the proposed solution.

To trace control-flow it is has been suggested that existing binaries should be go through an instrumentation process, where instructions which trigger transitions between BBLs are replaced with code which jumps execution to a trampoline (provided by the solution). This trampoline then sends the control-flow information to an application existing in the ``secure world'' which records said information. Execution is then transferred back to the original destination.

Additional details can be added to attestation reports via an api provided by the solution. This additional information may be important variables (which the developer of the application may which to keep track of) or environment information such as other processes running on the device. Allowing these additions represents an increase in functionality compared to standard attestation, as important information can be bound to a record of the prior control-flow.