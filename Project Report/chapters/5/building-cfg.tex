\section{Building CFGs}

\subsection{The problem behind building CFGs}

C-FLAT \cite{Abera2016} notes that efficiently computing a generic program's CFG, and accompanying exploration of all possible execution paths is an open problem. However it is also noted that static embedded application software is typically far simpler and it is therefore possible to reasonably compute a CFG.

\subsection{Existing tools}

\subsubsection{LLVM}

LLVM provides various methods of building CFGs which could be of use, for example:

\url{https://llvm.org/docs/Passes.html#dot-cfg-print-cfg-of-function-to-dot-file},\\
\url{https://llvm.org/docs/Passes.html#simplifycfg-simplify-the-cfg},\\
\url{https://llvm.org/docs/Passes.html#view-cfg-view-cfg-of-function}

Building CFGs in LLVM is not simple and is not a main feature of LLVM. The dot format is widely used in representing CFGs and further research is required to find whether it can be made to be compatible with control-flow tracing.

\subsubsection{Vulcan}

Vulcan \cite{Edwards2001} is used by \cite{Davi2012} and the seminal work \cite{Abadi2005}. Though it has since been taken out of production by Microsoft and is currently unavailable.

\subsubsection{Jakstab}

Jakstab \cite{Kinder2008},\cite{Kinder2010} can also be used to create a dot file containing the CFG of x86 executables, so may need enhancement to support instruction set architectures (ISAs) commonly used in embedded systems such as ARMv8-M or RISC-V. It was successfully used in \cite{Nguyen2013} to construct CFGs for x86 Complex Instruction Set Computer (CISC) instructions. The last development activity on this was 31st March 2017. The tool's author has been trying to improve on the tool with the aim of creating under-approximated CFGs rather than over-approximated ones (which can lead to vulnerabilities in resultant implementations) \cite{Kinder2012}.

\subsubsection{Others}

In C-FLAT \cite{Abera2016} the authors create their own analysis tool to achieve their branch-based requirements rather than create entire CFG models. They also use Capstone disassembly engine \footnote{\url{http://www.capstone-engine.org/}} however this is more for use dissasembling code rather than CFG building.

LiteHAX \cite{Dessouky2018} uses the `angr' \footnote{\url{http://angr.io/}} \cite{Shoshitaishvili2016} framework to generate the CFG outputting a networkx \cite{Hagberg2008} DiGraph. This project is still under active development. The output of this process would need to be compatible with the dynamic attestation. angr offers two types of computation methods - CFGFast (a static CFG) and CFGEmulated (a dynamic CFG).
During assessment of angr, utilising networkx utilities, we have found that only single edges are identified. For example ~\ref{lst:angr_result}

\begin{lstlisting}[caption={Example of results from angr CFG analysis of fauxware in form of .edgelist output from networkx utilities},label={lst:angr_result}]
<CFGNode rejected+0x16 0x400713[10]> <CFGNode 0x400570L[6]> {'jumpkind': 'Ijk_Call', 'ins_addr': 4196115, 'stmt_idx': 'default'}
<CFGNode main+0xa0 0x4007bdL[10]> <CFGNode accepted 0x4006ed[14]> {'jumpkind': 'Ijk_Call', 'ins_addr': 4196285L, 'stmt_idx': 'default'}
<CFGNode main+0xa0 0x4007bdL[10]> <CFGNode main+0xaa 0x4007c7L[2]> {'jumpkind': 'Ijk_FakeRet', 'ins_addr': 4196285L, 'stmt_idx': 'default'}
\end{lstlisting}

IDA Pro is a popular tool for reverse-engineering which is able to create CFGs, however due to its high licence cost we have been unable to evaluate its suitability. The high cost also prohibits its utilisation as part of the tool chain.

The recently released (March 2019) Ghidra \footnote{\url{https://ghidra-sre.org}} leverages CFGs to assist in carry out its core function as a reverse-engineering tool, and while we have not found a method of extracting the CFGs created, this may be possible in the future as development is ongoing.

Control-flow analysis techniques used by \cite{Abadi2005} are described in \cite{AhoAlfredV.2014C:pt},\cite{Atkinson},\cite{Wagner2002} which run at compile time. A list of tools used by the solution proposed in \cite{Abadi2005} are described in further detail in \cite{Edwards2001}(Vulcan), \cite{AhoAlfredV.2014C:pt},\cite{Feng2003} \cite{Gopalakrishna2005},\cite{Wagner2002}.