\section{Building CFGs}

\subsubsection*{The Problem Behind Generating CFGs}

Control-flow graphs (CFGs) are the cornerstone of control-flow integrity, and while several solutions already examined (for example, SEP \cite{Lee2019}, HAFIX \cite{Davi2015} and HCFI \cite{Danger2018}) do not require their construction, a CFG will be required when we consider audit as it provides a static reference point. Efficiently computing a generic program's CFG, and accompanying exploration of all possible execution paths is an open problem \cite{Abera2016}. However applications for embedded systems tend to be far smaller (and therefore fewer control-flow paths to map) than those operating on a general purpose computer, this means it is foreseeable that generation of a CFG for embedded systems applications should be possible.

\subsection{Tools For Generating CFGs}

A number of tools have been used in the CFI solutions already examined. These tools usually have a different primary function such as compilation or reverse-engineering, but offer CFG generation as a bi-product of the process required to provide their primary function. The following is a selection of tools already used by other CFI solutions.

\subsubsection*{LLVM}

LLVM \cite{LLVM:CGO04} consists of a collection of tools which have the primary function of compilation. It provides various methods of building CFGs, using different parts of some of the many toolchains. Unfortunately, building CFGs in LLVM is not as simple task, as this functionality is not a main feature. Some of the CFG methods included in LLVM create a CFG in the ``dot'' format, which is widely used in representing CFGs (The ``Jakstab'' tool also outputs generated CFGs in ``dot'' format \cite{Kinder2008}). If LLVM were to be used for generating CFGs, source code of an application would be required. This could prove problematic when adding the proposed solution to legacy systems. However, due to the prominence of LLVM, the CFG production functionality may be added into the compilation tool-chain, removing the need for extra steps to be undertaken by the software developer.

\subsubsection*{Vulcan}

Vulcan \cite{Edwards2001} is a far reaching project, designed to enable binary code modification, analysis and optimisation. It was used by \cite{Davi2012} and the seminal work \cite{Abadi2005}. It has since been taken out of production by Microsoft and is currently unavailable for further use or evaluation.

\subsubsection*{Jakstab}

Jakstab \cite{Kinder2008},\cite{Kinder2010}, like LLVM can be used to create a dot file containing the CFG. It is currently designed for use on x86 executables, so may need enhancement to support instruction set architectures (ISAs) commonly used in embedded systems such as ARMv8-M or RISC-V. It was successfully used in \cite{Nguyen2013} to construct CFGs for x86 Complex Instruction Set Computer (CISC) instructions. The last development activity on this was 31st March 2017. The tool's author has been trying to improve on the tool with the aim of creating under-approximated CFGs rather than over-approximated ones (which can lead to vulnerabilities in resultant implementations) \cite{Kinder2012}.

\subsubsection*{`angr' Framework}

LiteHAX \cite{Dessouky2018} uses the `angr' \cite{Shoshitaishvili2016} framework to generate the CFG outputting a networkx \cite{Hagberg2008} DiGraph. `angr' is primarily a reverse engineering tool, with the additional functionality of creating CFGs for target applications. It is notable that the framework is still under active development, meaning additional enhancements in the area of CFG generation may be added. `angr' offers two types of computation methods: \verb|CFGFast| (a static CFG generation method) and \verb|CFGEmulated| (a dynamic CFG generation method), which provide different levels of coverage for the resulting CFG.  

During assessment of angr using the sample `fauxware' binary included with the tool and through using networkx utilities, we have found that only single edges are identified, as shown in a sample of the output file at listing~\ref{lst:angr_result}. While this is useful for checking single transitions, the aim of the proposed solution is to track the entire path taken during execution, therefore we would require a list of valid paths rather than single edges.

\begin{lstlisting}[caption={An excerpt from the results of an angr CFG analysis of the file `fauxware'},label={lst:angr_result}]
<CFGNode rejected+0x16 0x400713[10]> <CFGNode 0x400570L[6]> {'jumpkind': 'Ijk_Call', 'ins_addr': 4196115, 'stmt_idx': 'default'}
<CFGNode main+0xa0 0x4007bdL[10]> <CFGNode accepted 0x4006ed[14]> {'jumpkind': 'Ijk_Call', 'ins_addr': 4196285L, 'stmt_idx': 'default'}
<CFGNode main+0xa0 0x4007bdL[10]> <CFGNode main+0xaa 0x4007c7L[2]> {'jumpkind': 'Ijk_FakeRet', 'ins_addr': 4196285L, 'stmt_idx': 'default'}
\end{lstlisting}

\subsubsection*{Alternative Tools}

IDA Pro \footnote{\url{https://www.hex-rays.com/products/ida/}} is a popular tool for reverse-engineering which is known to be able to create CFGs, however due to its high licence cost we have been unable to evaluate its suitability. The high cost also prohibits its utilisation as part of the tool chain.

The recently released (March 2019) Ghidra \footnote{\url{https://ghidra-sre.org}} leverages CFGs to assist in carrying out its core function as a reverse-engineering tool, and while we have not found a method of extracting the CFGs created, this may be possible in the future as development is ongoing.
