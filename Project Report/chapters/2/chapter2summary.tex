\section{Chapter Summary}
Embedded systems are a particular type of computer system which provide dedicated resources for specialist applications. Embedded systems often have small form factors, limited resources (such as less memory and slower processors providing only basic features) and are usually manufactured on a low budget. Due to these limitations they may not have fully functional operating systems, or may run specialist real-time operating systems. These properties mean that providing control-flow integrity (CFI) is not as easy at it is with general purpose computers, and due to the nature of the use of embedded systems (for example, personal medical equipment, sensors in critical systems and controllers for cyber-physical systems) the protection of CFI is extremely important.

Control-flow integrity (CFI) can be provided by a number of methods. One method is to closely bind the execution of an application with its control-flow graph (CFG). In order to do this a CFG needs to generated for an application and the execution of the application is controlled in a manner where it is compared against the generated CFG, if a violation is detected pre-determined actions are carries out. Another method is to build up a path of basic block transitions, where the signatures of each block is added to a function which creates a signature for the entire path taken. The resultant signature is then compared against a collection of valid pre-calculated signatures (created during analysis). Shadow stacks are presented as a method of protecting against attacks on the return addresses stored on the application's stack, which if successful would violate CFI.