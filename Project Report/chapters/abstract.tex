Control-flow integrity is a fundamental problem in computer security. Many attacks on software can be represented as a violation of a control-flow graph - the graphical representation of the legitimate control flow of an application. Attacks on hardware can be used to force the skipping of instructions, making the targeted application take a path which does not belong to the legitimate control-flow graph. Software-based attacks such as buffer overflows may target return addresses stored on an application's stack in order to divert its control-flow such that it does not follow the legitimate control-flow graph. Defences for attacks on control-flow attacks exist on general purpose computers, however embedded systems which have less resources than general purpose computers are not as well catered for. 