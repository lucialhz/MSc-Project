Control-flow integrity is a fundamental problem in computer security. Many attacks on software can be represented as a violation of a control-flow graph - the graphical representation of the legitimate control flow of an application. Attacks on hardware can be used to force the skipping of instructions, making the targeted application take a path which does not belong to the legitimate control-flow graph. Software-based attacks such as buffer overflows may target return addresses stored on an application's stack in order to divert its control-flow such that it does not follow the legitimate control-flow graph. Defences for attacks on control-flow attacks exist on general purpose computers, however embedded systems which have less resources than general purpose computers are not as well catered for in terms of control-flow integrity. Producing an audit trail of control-flow enables a verifier to gain assurances that applications have been executed in such a way that they legitimately followed their pre-defined control-flow graph.

Using control-flow integrity audit as a viewpoint this project surveys existing solutions in the area of control-flow integrity. Initially it undertakes a high-level review of secure software execution, associated subjects and the common properties for embedded systems. Following this it sets out an established criteria for the assessment of solutions providing static attestation of memory space and control-flow integrity protection and attestation, using this criteria these solutions are analysed and compared. The solutions presented consist of hardware-based solutions, where additional hardware compnonents or modifcations are required, and software-based solutions, where objectives are achieved using only software and existing Commercial off-the-shelf (COTS) hardware.
