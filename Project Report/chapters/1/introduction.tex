\section{Control Flow}
Control flow is how instructions within software applications transition from one instruction to another. Applications can be logically broken down into basic blocks (BBLs) through which the executing application traverses. The tracing of these transitions forms the control flow taken.

\subsection{Basic Blocks}
The principle of basic blocks creates a useful way of breaking down the instructions of an application into manageable chunks containing sequences of instructions. A basic block is a set of instructions which sequentially run from beginning to end. An example of this is illustrated in pseudo-code at listing~\ref{lst:basic_code}.

\begin{lstlisting}[language={[Sharp]C},caption={An example of code which can be broken down into basic blocks},label={lst:basic_code}]
void GreaterThanPlusOne(int firstNum, int secondNum)
    {
        string message;
        firstNum++;
        if (firstNum > secondNum)
        {
            message = "firstNum plus one is greater than secondNum";
        }
        if (firstNum <= secondNum)
        {
            message = "firstNum plus one less than or equal to secondNum"
        }
    }
\end{lstlisting}

The instructions from \verb|string message;| (line 3) up to and including \verb|if (firstNum > secondNum)| (line 5), \textit{basic block 1}, will execute sequentially (barring an exception being thrown). This is a basic block. The contents within the first \verb|if| (line 5) and also the second \verb|if| (line 9) parenthesis are also basic blocks, \textit{basic blocks 2 and 3} respectively.

So in the code described at listing~\ref{lst:basic_code}, the sequence of instructions can flow from the first basic block to either the second or the third. An unauthorised flow would be from basic block 2 to basic block 3 as there is no way that under normal operation both of these sequences of instructions would execute after one another.

Other examples of programming control flow statements which could form basic blocks are switch cases, loops, break statements and functions, as the these control the diversion of instruction execution depending on the current running circumstances (e.g. variable state).

\begin{figure}
  \centering
  \vspace*{0.5in}
  \includegraphics[scale=1]{cfg.png}
  \caption{The control-flow graph for code in listing~\ref{lst:basic_code}}
  \label{fig:basicCFG}
\end{figure}

\subsection{Control-flow graphs}
Control-flow graphs (CFGs) are the graphical representation of the series of valid transitions between basic blocks within an application (see figure~\ref{fig:basicCFG} for a CFG of listing~\ref{lst:basic_code}). It is conceivable that a CFG can become enormous as the size of an application grows, as the number of different valid paths along the graph will rapidly increase.
\ifnotesincluded
\furtherwork{(need to find reference for this)}
\fi. Valid control flow is where the flow which follows a CFG for a given application. Using figure \ref{fig:basicCFG} as an example for the code in listing \ref{lst:basic_code}, BBL 1 $\rightarrow$ BBL 2 and BBL 1 $\rightarrow$ BBL 3 are both valid flows, however BBL 1 $\rightarrow$ BBL 2 $\rightarrow$ BBL 3 is invalid. To create BBL 1 $\rightarrow$ BBL 2 $\rightarrow$ BBL 3 an attacker could change the value of \verb|secondNum| to one which is greater than or equal to \verb|firstNum| once the first \verb|if| statement has been passed. 

\subsection{Types of branches}
In order to understand control-flow it is important to understand the difference between the various branch types: direct, indirect, unintended and function returns.

\subsubsection{Direct}
A direct branch is one where the next instruction address is hard coded with the \verb|jump| instruction. This means the execution will flow to a pre-defined location.

A direct branch can be violated by hardware attacks such as glitching, which forces the skipping of instructions, or memory attacks, whereby the destination address is swapped for another, causing the execution to flow along an unintended path.

Attacks on direct branches can often be overlooked by control-flow integrity (CFI) protection mechanisms as the requirements of needing direct access to memory or performing glitching attacks often lie outside of the scope of these protections, as they view attacks on indirect branches and function returns as larger threats.

\subsubsection{Indirect}
An indirect branch is one where the next instruction address is set at run-time, either through registers, or pointers pointing to addresses in memory containing the \verb|jump| destination. This could be the result of a compiled \verb|switch|-\verb|case| statement or dynamic libraries (e.g. reflection). In the case of the \verb|switch|-\verb|case| statement it might not appear as indirect in the source code, however when compiled has the possibility of indirectness. Indirect branches are usually targets of buffer overflows.

\subsubsection{Unintended}
De Clerq \cite{DeClercq2017} describes an unintended branch as one which arises due to variable length instruction encoding in architectures. An attacker transfers control flow to the middle of an instruction. A high percentage (80\%) of gadgets exploit this, according to Kayaalp \cite{Kayaalp2014}. For more information see section 4.2 of Kayaalp's work \cite{Kayaalp2014} which describes unintended branches in detail. The threat of unintended branches is currently not such a problem in embedded systems as they tend to use fixed length instruction sets (such as RISC). However ARMv8, for example, has two sets of encoding which can be switched between, and as embedded systems' capabilities converge with those of general purpose computers we may see the introduction of variable length instruction encoding in the near future.

\subsubsection{Function Return}
A function returning to its caller is a branch in itself. When a function returns to its calling section of code the return address is retrieved from the stack and inserted in to the program counter (\verb|pc|). This presents a problem as the return address is not hard coded and according to a CFG, a function could legally return to any of the locations in code which call that function. Tracking the entire path taken along the CFG allows the checking of calling location, making sure the function has returned to the correct caller.

If the stack is attacked the return address can be re-written to point to existing code, or instructions placed in data (this is usually protected for by read $\oplus$ write). This is a well-explored area in computer security with various existing solutions including stack canaries, NX bit and Address Space Layout Randomisation (ASLR).

\ifnotesincluded

\subsubsection{Break statements}

\furtherwork{Should these be included?}

\fi

\subsection{Loops}
Loops are problematic in terms of CFGs. Modelling a loop which could run indefinitely is simply not feasible if full coverage is to be achieved. Various methods have been used to approach this, with some \cite{Abadi2005} simply compacting loops (this presents issues where an attacker can continually run a loop until they reach the desired outcome, such as performing an exhaustive search on passwords) while others \cite{Dessouky2017} track loop metadata such as iteration count for each unique path taken through a loop.