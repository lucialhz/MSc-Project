\section{Attacks on hardware}
\subsection{Introduction}

Attacks on hardware are amongst the hardest attacks to mitigate against, as having direct access to memory or even cache memory can enable an attacker to bypass many protections (for example software-based protections). Hardware attacks can be used to capture the leaking of cryptographic keys which presents a problem, particularly for attestation where a typical attestation report is signed using a signing key or a symmetric key is used for encrypting or applying a MAC to the attestation report.

\subsection{Invasive Attacks}
An invasive attack occurs when microprocessor removed and attacked directly through physical methods.In theory any microprocessor can be attacked in this way however does require expensive equipment and a large investment in time.

One example of an invasive attack is probing bus lines between blocks on a chip (with a hole being made in a chip's passivation layer). Here secret information is derived by observing information sent from one block to another. An extreme example of an invasive attack is using a focused ion beam to destroy or create tracks on the chip's surface. This could be used to reconnect disconnected fuses (this is a threat to the use of PUFs \furtherwork{what does it stand for, reference discussion on PUFs} where fuses are used to deactivate PUF derivation circuits). Fuses can also be used to turn off test modes which are used to read/write to memory addresses during manufacture, however this vulnerability has now been removed as test circuits are  removed from the chip when it is cut from the die during the manufacturing process \cite{Anderson1996} \cite{Kommerling1999}.

\subsection{Semi-Invasive Attacks}

Semi-invasive attacks require the surface of chip to be exposed, however the security is compromised without directly modifying the chip. Examples of semi-invasive attacks include observing electro-magnetic eminations using a suitable probe \cite{Gandolfi2007}, \cite{Quisquater2001}, injecting faults using laser \cite{Bar-el2006} or white light \cite{Skorobogatov2007}. Numerous more have been discussed in literature \cite{Skorobogatov2005}.

\subsubsection{Fault Injection}

Variations in supply voltage \cite{Anderson1996},\cite{Blomer2010} may cause processors to misinterpret or skip instructions, this is applicable to control-flow as the misinterpreting or skipping of instructions are attacks used to subvert the control-flow of applications.
Variations in external clock \cite{Anderson1996},\cite{Street},\cite{Kommerling1999} can cause data to be misread (i.e. data is attempted to be read before memory has time to latch-out correct value). This can lead to missing of instructions, which can be an attack vector if software is written in an insecure manner such as aborting an operation if an \verb|if| check is successful.
Extremes of temperature \cite{Boneh},\cite{Govindavajhala2003} can cause unpredictable effects in microprocessor. Two examples of effects obtained \cite{Bar-el2006} are random modification of RAM cells due to overheating and read write temperature thresholds in most NVM \furtherwork{what does this stand for?} not coinciding. In this case if the temperature is set to a level where write ops work but read do not a number of attacks can be mounted \furtherwork{such as?}. Laser light \cite{Habing1965} can be used where light arriving on a metal surface induces a current, which if intense enough could induce a fault in a circuit. White light \cite{Anderson1996} has been proposed as alternative to laser-based attacks  \cite{Skorobogatov2007}, however as it is not directional it is a challenge to apply to particular portions of microprocessor and therefore provide a targeted attack \furtherwork{pretty sure Mayes was successful with this}. Electromagnetic flux \cite{Samyde2003} has been used to change values in RAM, where strong eddy currents can affect microprocessors although this has only been observed in insecure microprocessors.

The effects of fault injection include reset data where data is forced to a blank state, data randomisation where data can be changed to a new random value and modifying of opcodes where instructions executed on chip's CPU are changes\cite{Anderson1996} which often has the same effect as previous examples but additionally allows for the removal of functions and breaking of loops. 

\furtherwork{Countermeasures given in \cite{Bar-el2006}}
  
\subsection{Non-Invasive Attacks}
Non-invasive attacks can be used to derive information without modification of hardware through information that leaks during computation of given command, or attempting to inject faults in manners other than light. Examples include monitoring power consumption \cite{Maurer1999}, \cite{MangardStefan2007Paa:} and injecting faults by glitching the power supply \cite{Anderson1996}, \cite{Bar-el2006}.

\section{Software attacks}
\subsubsection{Buffer overflow}

The aim of a buffer overflow attack is to manipulate control-flow information stored on the stack and heap of a program in order to achieve further objectives, examples of such information could be return addresses or other variables on which the basis of decisions are made (e.g. \verb|if| statements).

\subsubsection{Injected code}

When making use of injected code, control flow is deviated to existing injected code (usually as data). To mitigate against this NX bit can be implemented, this marks data memory as non-executable.

\subsubsection{Code re-use attack}
Existing code such as system functions or sequences of code which, when combined, have unintended consequences can be be the target of a deviated control-flow. See also return oriented programming and return-to-libc attacks.

\subsubsection{Return Oriented Programming (ROP)}
An attacker can string together (ordinarily benign) existing code sequences to form gadgets, which can result in malicious program actions. This is done through changing the return addresses on the stack to point to each snippet of the code sequence.

\subsubsection{return-to-libc}

An attacker replaces the return address on the stack to one which contains subroutines which already exist in memory such as functions which allow the execution of shell commands. \verb|libc| refers to the C standard library.

\subsubsection{Pointer subterfuge}

Pointer subterfuge is where the value of a pointer is modified by an attacker, there are various methods of achieving this and various points of attack. The simplest example is overrunning a buffer which is next to data in memory which is assigned to a pointer.

\subsubsection{Non-control data attacks}
These involve corrupting data which is used to decide on control flow, for example in a comparison in an \verb|if| statement. These usually produce unintended yet valid program flow, however there are examples which do not induce unintended flows (yet still allow an attacker to achieve their objectives), this is discussed in detail by Shacham \cite{Shacham2007}.