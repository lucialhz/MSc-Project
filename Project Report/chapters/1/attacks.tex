\UseRawInputEncoding
\section{Attacks on hardware} \label{hardwareAttacks}

Attacks on hardware are amongst the hardest attacks to mitigate against, as having direct access to memory or even cache memory can enable an attacker to bypass many protections (for example software-based protections). Hardware attacks can be used to capture the leaking of cryptographic keys, which presents a problem, particularly for attestation where a typical attestation report is signed using a signing key or when a symmetric key is used for encrypting or applying a MAC to the attestation report. The basis for this section stems from \cite{MayesKeithE2008SCTS} where attacks on hardware which are applicable to embedded systems are well explored.

\subsection{Invasive Attacks}
An invasive attack occurs when a microprocessor is removed and attacked directly through physical methods. In theory any microprocessor can be attacked in this way however it does require expensive equipment and a large investment in time.

One example of an invasive attack is probing bus lines between blocks on a chip (with a hole being made in a chip's passivation layer). Here, secret information is derived by observing information sent from one block to another. An extreme example of an invasive attack is using a focused ion beam to destroy or create tracks on the chip's surface. This could be used to reconnect disconnected fuses (this is a threat to the use of PUFs (see \ref{PUFExplaination} for more details) where fuses are used to deactivate PUF derivation circuits). Fuses can also be used to turn off test modes which are used to read\slash write to memory addresses during manufacture. Modern solutions avoid this vulnerability by removing test circuits from the chip when it is cut from the die during the manufacturing process \cite{Anderson1996},\cite{Kommerling1999}.

\subsection{Semi-Invasive Attacks}

Semi-invasive attacks  require the surface of chip to be exposed, however the security is compromised without directly modifying the chip. Examples of semi-invasive attacks include observing electro-magnetic eminations using a suitable probe \cite{Gandolfi2007},\cite{Quisquater2001}, injecting faults using laser \cite{Bar-el2006} or white light \cite{Skorobogatov2007}. Numerous more have been discussed in literature \cite{Skorobogatov2005}.

\subsubsection*{Fault Injection}

Variations in supply voltage may cause processors to misinterpret or skip instructions \cite{Anderson1996},\cite{Blomer2010}, this is applicable to control-flow as the misinterpreting or skipping of instructions are attacks used to subvert the control-flow of applications.
Variations in external clock can cause data to be misread \cite{Anderson1996},\cite{Street},\cite{Kommerling1999} (i.e. data is attempted to be read before memory has time to latch-out correct value). This can lead to the missing of instructions, which can be an attack vector if software is written in an insecure manner (such as aborting an operation if an \verb|if| check is successful).
Extremes of temperature can cause unpredictable effects in a microprocessor \cite{Boneh},\cite{Govindavajhala2003}. Two examples of these effects are random modification of RAM cells due to overheating and read write temperature thresholds in most non-volatile memory (NVM) not coinciding \cite{Bar-el2006}. In this case if the temperature is set to a level where write ops work but read do not a number of attacks can be mounted. Laser light can be used where light arriving on a metal surface induces a current, which if intense enough could induce a fault in a circuit \cite{Habing1965}. White light has been proposed as alternative to laser-based attacks \cite{Anderson1996},\cite{Skorobogatov2007}, however as it is not directional it is challenging to focus on a particular areas of a microprocessor and therefore provide a targeted attack. Electromagnetic flux has been used to change values in RAM, where strong eddy currents can affect microprocessors although this has only been observed in insecure microprocessors \cite{Samyde2003}.

\begin{lstlisting}[language={[Sharp]C},caption={An example of code which is vulnerable to skipping of instructions},label={lst:instructionSkip}]
int CheckNumberOfAttempts(int attemptCount, int attemptLimit)
{
    if (attemptCount => attemptLimit)
        return -1;
    ContinueLogin();
    return 1;
}
\end{lstlisting}

The effects of fault injection include resetting data where data is forced to a blank state, data randomisation where data can be changed to a new random value and modifying of opcodes where instructions executed on a chip's CPU are changed which often has the same effect as previous examples but additionally allows for the removal of functions and breaking of loops \cite{Anderson1996}.
  
\subsection{Non-Invasive Attacks}
Non-invasive attacks can be used to derive information without modification of hardware through information that leaks during computation of a given command, or attempting to inject faults in manners other than light. Examples include monitoring power consumption \cite{Maurer1999},\cite{MangardStefan2007Paa:} and injecting faults by glitching the power supply \cite{Anderson1996},\cite{Bar-el2006}.

\section{Software attacks}

Attacks on software represent a likely threat to the security of embedded systems as they can be carried out remotely or simply stumbled upon (fuzzed). Once an exploit is discovered it is highly repeatable and can be crafted into a number attack methods. Embedded systems can be more vulnerable than other computer systems due to the language used to write applications - C. As C provides the developer with memory operations but, as a result, also presents the attacker with an opportunity to exploit any bugs in the code, usually by manipulating memory.

\subsection{Common software attacks methods}

The aim of a \textbf{buffer overflow attack} is to manipulate control-flow information stored on the stack and heap of a program in order to achieve further objectives, examples of such information could be return addresses or other variables on which the basis of decisions are made (e.g. \verb|if| statements).

When performing an \textbf{injected code attack}, control flow is diverted to existing injected code (usually as data). To mitigate against this, NX bit can be implemented which marks data memory as non-executable.

Existing code such as system functions or sequences of code which, when combined, have unintended consequences can be the target of a deviated control-flow. This is known as a \textbf{code re-use attack}. See also return oriented programming and return-to-libc attacks.

An attacker can string together (ordinarily benign) existing code sequences to form gadgets, which can result in malicious program actions. This attack is referred to as \textbf{return-oriented programming (ROP)}. This is done through changing the return addresses on the stack to point at each snippet of the code sequence.

An attacker replaces the return address on the stack to one which contains subroutines which already exist in memory (\textbf{return-to-libc}) such as functions which allow the execution of shell commands. \verb|libc| refers to the C standard library.

\textbf{Pointer subterfuge} is where the value of a pointer is modified by an attacker, there are various methods of achieving this and various points of attack. The simplest example is overrunning a buffer which is next to data in memory which is later assigned to a pointer as shown in listing \ref{lst:pointerSubterfuge} \cite{Wang2016a}.

\begin{lstlisting}[language={C},caption={An example of code which is vulnerable to pointer subterfuge \cite{Wang2016a}},label={lst:pointerSubterfuge}]
int main(){
  int *ptr = ...;
  int val = ...;
  char buf[16];

  gets(buf);

  *ptr = val;
  return 0;
}
\end{lstlisting}


\textbf{Non-control data attacks} involve corrupting data which is used to decide on control flow, for example in a comparison in an \verb|if| statement. These usually produce unintended yet valid program flow, however there are examples which do not induce unintended flows (yet still allow an attacker to achieve their objectives), this is discussed in detail by Shacham \cite{Shacham2007}.