\section{Problem Description}
\subsection{Introduction}

CFI has been a persistent problem in computer security, with examples of attacks including return oriented programming, stack overflows and hardware-based attacks (see \ref{hardwareAttacks}). Several means of defence have been designed to address some or all of the problematic results (see \ref{solutionsDescriptions}). These include:

\subsubsection*{Prevention}
Where violation of the control-flow integrity is prevented before it occurs. Methods include:
\begin{itemize}
	\item Read $\oplus$ Write memory, which prevents the execution of data memory and the overwriting of memory containing executable instructions.
	\item Encrypted instructions, which enforce decryption based on a particular order of instructions following the application CFG.
\end{itemize}

\subsubsection*{Detection}
Where processes are put in place to detect a violation of control-flow integrity and act upon detection, in this instance the intention is to catch the violation before too much damage has been done. Methods include:
\begin{itemize}
	\item Stack canaries, which detect ROP attempts therefore preventing violation of CFI.
	\item Shadow stacks, which aid in preventing return-oriented programming by detecting ROP attempts and assist in preventing violation of CFI
	\item Software-based CFI, which adds checking of calling IDs between each basic block to ensure they are permitted within the CFG. This was first discussed by Abadi et al. \cite{Abadi2005} and has been subject to a large amount of subsequent research.
\end{itemize}

\subsubsection*{Attestation}
Where a third party (verifier) receives a report, which could only have come from the device (prover), stating either that the control-flow integrity has not been violated or describing the taken control-flow path (therefore allowing the verifier to compare against the CFG to be sure that control-flow integrity has been maintained). Example methods include:
\begin{itemize}
	\item Basic-block IDs are hashed to create a hash chain representation of the control-flow path taken.
	\item Instructions are intercepted by secondary hardware monitor to take flow measurements, comparing these to the valid instructions as described by the CFG.
\end{itemize}

\subsubsection*{Data-Flow Integrity}

Data flow integrity \cite{Castro2006} is a measure used to ensure that data only flows in the correct direction from authorised callers. This can be used to ensure only the correct caller has changed an important variable - for example one which a decision is based on (e.g. inside an \verb|if| statement), this can also be used to protect the stack.
In  \cite{Castro2006} the solution maintains a table for each piece of data accessed with the address of last accessor. If the last accessor is not within the legal list a security failure exists and an exception is thrown.

Dynamic taint analysis \cite{Davi2009} is another method used to detect and prevent control-data and non-control-data attacks by labelling data as untrusted and tracking that data to make sure it is not used to determine program execution. With dynamic taint analysis there is no need for source code, however it does yield false positives, has a high overhead and requires additional hardware support as noted by Castro et al. \cite{Castro2006}.

\subsection{Auditing Control-Flow}
CFI prevention and detection has proven to lead to a slight computational overhead \cite{DeClercq2017} and a large storage overhead (if the CFG were to be stored locally), while attestation requires an always-on connection. What if we can store the control flow of an application on a hard disk, which can be retrieved later? This could be useful to prove to a third party that a disputed result of a computation is correct (or at least the steps taken during the computation were correct), or providing a record of the control-flow which has been taken during the execution of a  vitally important operation.

Keeping audit records of control-flow could also be useful for spot-checking inaccessible devices which have no internet connection, such as industrial control systems (ICS) components or medical devices which can only be communicated with during medical check-ups.

So, while we would not provide immediate detection or real-time attestation, the contents of the audit file can be compared with the CFG for an application at a later time. This could also be useful for reducing computing requirements as comparisons could be appropriately scheduled for the efficient use of processor time, rather than being performed on an ad-hoc basis (as required by some attestation methods).
