\section{Security Analysis}

The solution presented fits with the security requirements in the following ways:
\begin{enumerate}
	\item \textbf{Works with compiled binaries}: The solution works with compiled binaries, where the binaries go through an instrumentation process.
	\item \textbf{Works with external binaries}: As with compiled binaries, external binaries will have to undergo instrumentation.
	\item \textbf{Works with embedded systems}: ARM TrustZone has been implemented on Cortex M processors which are widely used in embedded systems. Key storage and external storage will need to be in place for an embedded system to make use of this solution.
	\item \textbf{Granularity}: The solution provides granularity down to a basic block (BBL) level.
\end{enumerate}

Obvious vulnerabilities in the solution are the breach of the `secure world' where the measurement engine is manipulated to create `good' attestation files, deletion of files and resulting loss of audit capabilities (a backup method would somewhat alleviate this problem) and cloning of audit files. Cloning of audit files represents a significant problem due to the fact that this is an offline process, so a method will need to be used to ensure freshness of files, for example a tamper-proof date-time stamp.

Granularity to an instruction level can only be provided by a hardware solution which intercepts each instruction.