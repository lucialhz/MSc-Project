\section{Security Analysis}

The solution presented fits with the security requirements in the following ways:
\begin{enumerate}
	\item \textbf{Works with compiled binaries}: The solution works with compiled binaries, where the binaries go through an instrumentation process.
	\item \textbf{Works with external binaries}: As with compiled binaries, external binaries will have to undergo instrumentation.
	\item \textbf{Works with embedded systems}: ARM TrustZone has been implemented on Cortex M processors which are widely used in embedded systems. Key storage and external storage will need to be in place for an embedded system to make use of this solution.
	\item \textbf{Granularity}: The solution provides granularity down to a basic block (BBL) level.
\end{enumerate}

Obvious vulnerabilities in the solution are the breach of the `secure world' where the measurement engine is manipulated to create `good' attestation files, deletion of files and resulting loss of audit capabilities (a backup method would somewhat alleviate this problem) and cloning of audit files. Cloning of audit files represents a significant problem due to the fact that this is an offline process, so a method will need to be used to ensure freshness of files, for example a tamper-proof date-time stamp.

Proxy-based attacks (as described in \cite{Seshadri2007} and \cite{Li2011}) as well as replay or cloning attacks are a threat to a simple implementation of the solution. The inclusion of an anti-tamper real time clock (similar to the one provided by the Java iButton described in \cite{Chong2003}) for use in signing or date-stamping the audit records would mitigate against this vulnerability.

Key management has been an open problem in the field of embedded systems, where the use of a TPM is not a given. The secure storage of a cryptographic key is vital for providing audit capabilities, as the signing of or applying a MAC to an audit record is vital from proving its origin.

Granularity to an instruction level can only be provided by a hardware solution which intercepts each instruction.

\subsubsection{Hardware Implementation}
Using a hardware solution it would be possible to address further criteria which the software-based solution was unable to address, much like the improvements LO-FAT\cite{Dessouky2017} made on C-FLAT \cite{Abera2016}. By moving monitoring and auditing capabilities to additional hardware such as a co-processor, or dedicated circuitry it would be possible to remove the majority of the overhead introduced when monitoring control-flow. Dedicated hardware with the ability to intercept instructions as they flow through the processing pipeline would enable monitoring of control-flow to a deeper granularity, where individual instructions of a basic block could be tracked in the form of a hash and number of instructions present within the basic block, this would be provided in a similar fashion as CCFI-Cache \cite{Danger2018}.

